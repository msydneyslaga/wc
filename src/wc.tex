\documentclass[12pt]{article}
\usepackage{graphicx} % Required for inserting images
\usepackage{listings}
\usepackage{fontspec}
\usepackage[dvipsnames]{xcolor}
\usepackage[outputdir=build]{minted}
% \usepackage{minted}
\usepackage[fontsize=16pt]{fontsize}

\usepackage{geometry}
\geometry{legalpaper, portrait, margin=2cm}

\usemintedstyle{emacs}

\setmonofont{VictorMonoNF}
[ Extension = .ttf
, UprightFont = res/*-Regular
, ItalicFont = res/*-Italic
, BoldFont = res/*-SemiBold
, SwashFont = res/*-SemiBold
]

% \newenvironment{code}
%     {\begin{minted}[frame=leftline,fontsize=\small]{haskell}}
%     {\\\\\end{minted}}
\newenvironment{code}{\VerbatimEnvironment\begin{minted}[frame=leftline,fontsize=\footnotesize]{haskell}}{\end{minted}} 

\definecolor{greybg}{rgb}{0.95,0.95,0.95}

\title{\vspace{-2.5cm}AP CSP Create Task}
\author{Madeline Sydney Slaga}
\date{February 2023}

\begin{document}

\maketitle
comment comment
\begin{code}
-- File IO
import System.IO

-- Command-line argument handling
import System.Environment (getArgs)
import System.Console.GetOpt

-- Very convenient functions which reduce verbosity
import Data.Maybe (fromMaybe)
import Control.Monad (forM_)
import Text.Printf (printf)

-- Used for counting bytes, as the `Char` type can contain Unicode
-- Import qualified because of conflicting identifiers
import qualified Data.ByteString as BS (length)
import qualified Data.ByteString.UTF8 as UTF8 (fromString)
\end{code}

\section{Command-line Options}
comment
\begin{code}
data Flag
    = Version
    | CountBytes
    | CountLines
    | CountWords
    | CountCharacters
    deriving (Show, Eq)
\end{code}

\begin{code}
options :: [OptDescr Flag]
options =
    [ Option ['V'] ["version"] (NoArg Version)
        "Display version number"
    , Option ['c'] [ ] (NoArg CountBytes)
        "The number of bytes in each file is written to stdout"
    , Option ['l'] [ ] (NoArg CountLines) 
        "The number of lines in each file is written to stdout"
    , Option ['w'] [ ] (NoArg CountWords) 
        "The number of words in each file is written to stdout"
    ]
\end{code}

\end{document}


